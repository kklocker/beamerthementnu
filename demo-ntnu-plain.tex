\documentclass[aspectratio=169]{beamer}
\usepackage[english]{babel}
\usepackage{booktabs,listings}
\usepackage[T1]{fontenc}
\usepackage[utf8]{inputenc}
\lstset{basicstyle=\ttfamily}
\setlength{\parskip}{.5\baselineskip}
\usetheme[slogan=english, style=plain,mathfont=serif]{NTNU}
%
% Edit your meta data here
%
	\title{\LaTeX{}-Beamer Style for NTNU}
	\subtitle{Demo of the  \texttt{plain} style}
	\author{Ronny Bergmann}
	\date{\today}

\begin{document}
	\maketitle
    \begin{frame}[fragile]{A slide in \texttt{plain} style}
        This style is set up with

        \lstinline!\usetheme[slogan=english, style=plain]{NTNU}!

        Here you can see that the \texttt{plain} (default) style has just a logo in the bottom left and a slide number in the bottom right.

        The slogan can be english or norwegian depending on the \lstinline!slogan=! option. Here it is set to \lstinline!english! (default).

        The slide number can also have the total number of slides setting \lstinline!frametotal=true!, which is \lstinline!false! (the default) here.
    \end{frame}
\end{document}
